\chapterinitial{The introduction}
New pervasive technologies in the workplace have been introduced to enhance productivity (e.g., a tool that provides an aggregated productivity score based on, for example, email use on the move, network connectivity, and exchanged content). Yet, some of them are judged to be invasive so much so that they make it hard to build a culture of trust at work, and often impacting workers' productivity and well-being in negative ways~\cite{alge2013workplace}. While these technologies hold the promise of enabling employees to be productive, report after report has highlighted the outcries of AI-based tools being biased and unfair, and lacking transparency and accountability~\cite{buolamwini2018gender}. Systems are now being used to analyze footage from security cameras in workplace to detect, for example, when employees are not complying with social distancing rules\footnote{\url{https://www.ft.com/content/58bdc9cd-18cc-44f6-bc9b-8ca4ac598fc8}}; while there is a handful of good intentions behind such a technology (e.g., ensuring safe return to the office after the COVID-19 pandemic), the very same technology could be used to surveil employees' movements, or time away from desk. Companies now hold protected intellectual properties on technologies that use ultrasonic sound pulses to detect worker's location and monitor their interactions with inventory bins in factories.\footnote{Wrist band haptic feedback system: \url{https://patents.google.com/patent/WO2017172347A1/}}

As we move towards a future likely ruled by big data and powerful AI algorithms, important questions arise relating to the psychological impacts of surveillance, data governance, and compliance with ethical and moral concerns (\url{https://social-dynamics.net/responsibleai}). To make the first steps in answering such questions, we set out to understand how employees judge pervasive technologies in the workplace and, accordingly, determine how desirable technologies are supposed to behave both onsite and remotely. In so doing, we made two sets of contributions: First, we considered 16 pervasive technologies that track workplace productivity based on a variety of inputs, and conducted a study in which 131 US-based crowd-workers judged these technologies along the 5 well-established moral dimensions of harm, fairness, loyalty, authority, and purity~\cite{haidt2007new}. We found that the judgments of a scenario were based on specific heuristics reflecting whether the scenario: was currently supported by existing technologies; interfered with current ways of working or was not fit for purpose; and was considered irresponsible by causing harm or infringing on individual rights. Second, we measured the moral dimensions associated with each scenario by asking crowd-workers to associate words reflecting the five moral dimensions with it. We found that morally right technologies were those that track productivity based on task completion, work emails, and audio and textual conversations during meetings, whereas morally wrong technologies were those that involved some kind of body-tracking such as tracking physical movements and facial expressions.
