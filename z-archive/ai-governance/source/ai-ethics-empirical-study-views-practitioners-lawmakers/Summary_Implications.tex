\section{Key findings and implications}
We now outline the key findings of the study to answer the RQs - AI ethics principles, challenges, severity impact and the statistically significant differences between the perceptions of practitioners and lawmakers. We also report the research and practical implications of the study findings.

\subsection{Summary and interpretation of the key findings}
The results of each RQ are thoroughly discussed in Section \ref{sec:Results} and the summary of core findings is presented in Table \ref{tab:Summary of key findings}, addressing RQ1 - confirming the identified AI ethics principles and challenges, RQ2 - measuring the severity impacts of the challenges across principles and RQ3 - practitioners and lawmakers perceptions of AI ethics principles and challenges. For RQ1, the summary of the study results highlights that both practitioners and lawmakers empirically confirm the AI ethics principles and challenges identified in our recent SLR study \cite{AR13}. We noticed that ($\geq$ 60\%) of the survey participants agreed to consider the reported principles and challenges. They further define the ranks of identified principles and challenges across a five-point Likert scale, which indicates that \textit{transparency}, \textit{accountability} and \textit{privacy} are the most critical principles \cite{AR15}\cite{AR10}; on the other hand, \textit{lack of ethical knowledge}, \textit{no legal frameworks} and \textit{monitoring bodies} appeared as the most frequently occurred challenging factors. Similarly, the summary of findings to address RQ2 reveal that \textit{conflict in practice} emerged as the most severe challenging factor for the identified AI ethics principles (see Table \ref{tab:Summary of key findings}). For certain cases, the AI ethics principles come into conflict, and their practical values become unrealistic- prioritizing one might inadvertently compromise another \cite{whittlestone2019role}. Whittlestone et al. \cite{whittlestone2019role} argue that thorough exploration is required to encounter and articulate the conflict and tensions across the AI ethics principles. For RQ3, we noticed that the perceptions of both types of populations (practitioners, lawmakers) are correlated and statistically significant for specific challenges. It is because the existing principles are too vague, generic, conceptual and no match for the specific and complex AI problems. Stakeholders have different perceptions of the challenges raised because of implementing the generic AI ethics principles. A broader consensus of multiple stakeholders- practitioners, lawmakers, and regulatory bodies required to define domain-specific principles and guidelines \cite{whittlestone2019role}. Overall, the summary of the findings in Table \ref{tab:Summary of key findings} is self-explanatory and encapsulates the core results discussed in Section \ref{sec:Results}.

\begin{table*}
% \scriptsize
\centering
\caption{Summary of the key findings}
\label{tab:Summary of key findings}
% \resizebox{\textwidth}{!}{%
\begin{tabular}{|m{3cm}|m{10cm}|}
\hline
\textbf{Research Question} & \textbf{Summary} \\ \hline
RQ1: What are the practitioners' and lawmakers' insights on AI ethics principles and challenges? & 
\begin{itemize}%[itemsep=0pt, topsep=0pt, parsep=2pt]
    \item Practitioners and lawmakers empirically confirm the principles and challenges reported in our previously published SLR study \cite{AR13}.
    \item \textit{Transparency}, \textit{accountability}, and \textit{privacy} emerged as the most perceived principles.
    \item \textit{Lack of ethical knowledge}, \textit{no legal frameworks}, and \textit{lacking monitoring bodies} appeared to be the most frequently cited challenging factors.
\end{itemize} \\
\hline
RQ2: What would be the severity impacts of identified challenges across the AI ethics principles? & 
Transparency \vspace{0mm}
\begin{itemize}%[itemsep=0pt, topsep=2pt, parsep=2pt]
    \item \textit{Interpret principles differently} has long-term impacts (50\% major, 27\% catastrophic)
    \item \textit{Lack of ethical knowledge} has short-term impacts (7\% insignificant, 25\% minor, 20\% moderate)
\end{itemize} 
Privacy \vspace{0mm}
\begin{itemize}%[itemsep=0pt, topsep=2pt, parsep=2pt]
    \item \textit{Conflict in practice} has long-term impacts (53\% major, 21\% catastrophic)
    \item \textit{Lack of technical understanding} has short-term impacts (8\% insignificant, 14\% minor, 23\% moderate)
\end{itemize} 
Accountability \vspace{0mm}
\begin{itemize}%[itemsep=0pt, topsep=2pt, parsep=2pt]
    \item \textit{Extra constraints} has long-term impacts (53\% major, 29\% catastrophic)
    \item \textit{Lack of ethical knowledge} has short-term impacts (6\% insignificant, 16\% minor, 18\% moderate)
    \item \textit{Lack of technical understanding} has short-term impacts (1\% insignificant, 18\% minor, 21\% moderate)
\end{itemize} 
Fairness \vspace{0mm}
\begin{itemize}%[itemsep=0pt, topsep=2pt, parsep=2pt]
    \item \textit{Extra constraints} has long-term impacts (49\% major, 33\% catastrophic)
    \item \textit{Interpret principles differently} has short-term impacts (3\% insignificant, 17\% minor, 24\% moderate)
\end{itemize} 
Autonomy \vspace{0mm}
\begin{itemize}%[itemsep=0pt, topsep=2pt, parsep=2pt]
    \item \textit{Conflict in practice} has long-term impacts (59\% major, 20\% catastrophic)
    \item \textit{Lack of ethical knowledge} has short-term impacts (6\% insignificant, 13\% minor, 24\% moderate)
\end{itemize} 
Explainability \vspace{0mm}
\begin{itemize}%[itemsep=0pt, topsep=2pt, parsep=2pt]
    \item \textit{Conflict in practice } has long-term impacts (55\% major, 24\% catastrophic)
    \item \textit{Lack of ethical knowledge} has short-term impacts (2\% insignificant, 12\% minor, 27\% moderate)
\end{itemize} 
\\
\hline
RQ3: How these challenges and principles are differently perceived by practitioners and lawmakers? &
\begin{itemize}%[itemsep=0pt, topsep=2pt, parsep=2pt]
    \item The perceptions of practitioners and lawmakers regarding AI ethics principles are strongly correlated (rs=0.819, p=0.000)
    \item The perceptions of practitioners and lawmakers regarding AI ethics challenges are moderately correlated (rs=0.628, p=0.012)
\end{itemize} \\
\hline
\end{tabular}%
% }
\end{table*}

\subsection{Research implications}
\begin{itemize}
    \item We found that most survey respondents agreed to consider the reported principles of AI ethics; however, \textit{transparency}, \textit{accountability} and \textit{privacy} are identified as the most common principles. The study findings complement the existing literature by revealing the most critical principles and call for future research to define the best solutions for scaling the highly significant principles in practice \cite{morley2020initial}.
    \item Regarding the challenges of AI ethics, the survey results confirm the findings of our recent SLR study \cite{AR13}, and determine \textit{lack of ethical knowledge}, \textit{no legal frameworks} and \textit{lacking monitoring bodies} as the high-ranked barriers. The identified challenges are core focus areas that need further research to explore the root causes and best practices to mitigate them.
    \item The study findings indicate that \textit{conflict in practice} is the most severe challenge of AI ethics principles. It opens the door for action-guiding future research - AI ethics principles must be contextualized to balance the conflicts \cite{whittlestone2019role}. The principles need to structure as standards, regulations and codes to resolve the conflicting tensions \cite{whittlestone2019role}.
\end{itemize}

Overall, the study findings complement the emerging research on AI ethics, particularly recognizing the perceptions of two different types of populations (practitioners and lawmakers). Researchers can quickly look up the study results and develop new hypotheses to streamline the mentioned gaps, e.g., solutions to scale the identified principles in practices, explore the causes and mitigation practices of reported challenges, and tailor the existing principles to fit in specific scenarios.
\subsection{Practical implications}
\begin{itemize}
    \item The study findings provide an overview of AI ethics principles, challenges, and practitioners can consider the overall understanding of study findings for defining ethically mature AI processes.
    \item Manifesting AI ethics principles in practice is hard because of various challenging factors. However, we measured the erroneous impacts of these challenges - revealing the most severe barriers practitioners need to tackle before embarking on ethics in AI. 
    \item In general, the study results can facilitate practitioners to get an overview and analyze the extent to which the reported principles and challenges can be leveraged to support AI ethics in the industrial setting.
 \end{itemize}
   
AI ethics in practice is still a widely unexplored research area. We invite researchers from academia and practitioners from industry to jointly contribute by sharing their experiences and to present potential solutions for AI ethics problems. This effort will bridge the gap between academia and industrial practices.  

