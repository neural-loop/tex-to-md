\section{Related Work}
We review the most relevant existing work classified into  two categories, focused on i) AI ethics principles and guidelines, and ii) AI ethics frameworks. A conclusive summary at the end position the scope and contributions of the proposed study.

\subsection{AI ethics principles and guidelines }
Lu et al. \cite{lu2022software} interviewed 21 practitioners and verified that the existing AI ethics principles are broad and do not provide tangible guidance to develop ethically aligned AI systems. Their study findings uncover the fact that AI ethics practices are often considered ad hoc and ignored for continuous learning. Based on the interview findings, Lu et al. \cite{lu2022software} proposed a list of patterns and processes which can be embedded as product features to design a responsible AI system. The proposed design patterns are mainly used to support the core AI ethics principles mentioned by the interview participants. Similarly, Lu et al. \cite{lu2022towards} also conducted an SLR study and defined a software engineering roadmap to develop AI systems. The proposed roadmap covers the entire process life cycle focusing on responsible AI governance, defines process-oriented practices, and presents architectural patterns, styles and methods to build responsible AI systems by design.

Vakkuri et al. \cite{vakkuri2022software} conducted an industrial survey with 249 practitioners to understand and verify the mentioned research gap based on the EU AI ethics guidelines \cite{AR2}. The survey results highlight that most of the companies ignored considering the societal and environmental requirements for developing AI systems. Moreover, the surveyed participants largely considered the product customers as the only stakeholders of AI ethics perspectives; however, it is more narrow in the AI domain, covering customers, regulatory bodies, practitioners, and society. Consequently, the focus should be on multiple AI ethics principles, e.g., \textit{accountability}, \textit{responsibility}, and \textit{transparency}.

Ibanez and Olmeda \cite{ibanez2021operationalising} conducted semi-structured interviews with 22 practitioners and two focus groups to know how software development organizations address ethical concerns in AI systems. The interview findings raised various issues related to AI ethics principles and practice including \textit{governance}, \textit{accountability}, \textit{privacy}, \textit{fairness}, and \textit{explainability}. Moreover, the interview participants provide some suggestions to operationalize AI ethics, e.g., promoting domain focus standardization, embracing data-driven organizational culture, presenting a particular code of ethics and fostering AI ethics awareness. In conclusion, Ibanez and Olmeda \cite{ibanez2021operationalising} called for a set of actions to distinguish project stakeholders, develop a socio-technical project team, and regularly evaluate the AI projects practices, processes and policies.

\subsection{AI ethics frameworks}
Vakkuri et al. \cite{vakkuri2021eccola} developed the ECCOLA framework to provide a tool for implementing ethics in AI. ECCOLA aims to assist practitioners, and AI-specific software development organizations in adopting ethically aligned development processes. The proposed framework supports iterative development and consists of a deck of cards (modules) that could be tailored to a specific context. The card mainly defines various themes of AI ethics, which were identified in the existing AI ethics guidelines. The ECCOLA framework is evaluated in both the academia and industrial domain to understand its real-world implications and limitations. 

Floridi et al. \cite{floridi2018ai4people} proposed the AI4People framework comprising five principles and twenty recommendations to lay the foundation for “Good AI Society”. The available sets of AI ethics principles are comparatively synthesized to understand the commonalities and significant differences. The comparison findings reveal four AI ethics principles (\textit{beneficence}, \textit{non‑maleficence}, \textit{autonomy}, \textit{justice}) with a new additional principle  (\textit{explicability}) to structure the AI4People framework. Finally, twenty action points were devised to scale the mentioned principles in practice. The overall aim of the proposed framework is to move the dialogue forward from theoretical principles to in-action policies. Such policies shield human autonomy, increase social empowerment, and decrease inequality.

Leikas et al. \cite{leikas2019ethical} presented a framework that focuses on ethics by design in decision-making systems. The current design approaches, practices, theories and concepts of autonomous intelligent systems are reviewed to structure the proposed ethical framework. The framework could use to recommend a set of AI ethics principles and practices for a specific scenario. The framework captured the human-centric details of a particular case study and used the details to identify the ethical requirements of concerned stakeholders and transfer them to the design goals. Leikas et al. \cite{leikas2019ethical} called for future studies to evaluate the real-world significance of the proposed framework in industrial scenarios.

\subsection{Conclusive summary}
The reported studies \cite{lu2022software}\cite{lu2022towards}\cite{vakkuri2022software}\cite{ibanez2021operationalising} are grounded on empirical findings and fine-granular analysis of extant AI ethics principles and guidelines. To complement empiricism in exploring AI ethics principles and challenges, this study explicitly analyzed and discussed the principles based on the perceptions of two different types of populations (practitioners and lawmakers). Studies \cite{vakkuri2021eccola}\cite{floridi2018ai4people}\cite{leikas2019ethical} are conducted to  design various frameworks to operationalize the AI ethics principles; however, no research has yet been done to streamline the plethora of challenges in adopting the widely defined principles and frameworks. Our study preliminarily focused on survey-driven validation of AI ethics principles and challenges by practitioners and lawmakers to complement the body of research comprising the recent industrial studies on AI ethics principles \cite{lu2022software}\cite{lu2022towards}\cite{vakkuri2022software}\cite{ibanez2021operationalising} and frameworks \cite{vakkuri2021eccola}\cite{floridi2018ai4people}\cite{leikas2019ethical}.