
\section{Threats to validity}
\label{threats to validity}
Various threats could affect the validity of the study findings. We followed the guidelines presented in \cite{easterbrook2008selecting} and categorized the potential threats across the following four different types:
\subsection{Internal validity}
Internal validity refers to particular factors that impact methodological rigor. The first internal validity threat in this study is the understandability of the survey instrument. The survey participants might have a different understanding of the survey content; however, the survey instrument was piloted based on the expert's opinions to improve the readability and understandability of the questions (see Section \ref{SettingtheStage}). Moreover, the domain expertise of survey participants could be a potential internal validity threat. We tried to mitigate this threat by exploring various social media networks and used personal links to approach the most suitable candidates. Furthermore, we explicitly mentioned the characteristics of prospective participants in the survey information sheet \cite{replication}. The interpersonal bias in the data collection and analysis process could threaten the internal validity of study findings. However, the survey data is collected, analyzed, organized and reported based on the final consensus made by all the authors (see Section \ref{SettingtheStage} and Section \ref{sec:Results}). 
\subsection{Construct validity}
Construct validity is the extent to which the study constructs are well-interpreted and defended. In this study, AI ethics principles and challenges are the core constructs. The reliability and authenticity of the selected data sources (platforms) is a possible construct validity threat. This threat has been alleviated by searching social media and professional research networks to identify the most relevant groups or individuals. We thoroughly read the group discussions to ensure that the group members mainly discussed AI ethics issues. Similarly, we explored the profile details and interests of the targeted individuals. 
\subsection{External validity}
External validity is the extent to which the study findings based on a particular data sample could be broadly generalized to other contexts. The survey sample size might not be representative to provide a concrete foundation for generalizing the study findings. However, we received 99 valid responses from 20 countries across 5 continents, having a diverse range of experiences, working in various domains on distinct roles in different size organizations (see Figure \ref{Fig:Demographics}). We concede that the study findings could not be generalized at a large scale or consider the identified principles and challenges for all types of AI systems. However, considering the demographic details of the survey respondents (see Figure \ref{Fig:Demographics}), we believe that the study results can support the overall generalizability to some extent.   
\subsection{Conclusion validity}
Conclusion validity is the extent to which certain factors affect valid conclusions in empirical research. To lessen this threat, the first two authors mainly participated in the data collection process; however, the next authors participated in the consent meetings to share feedback and review the survey activities (see Section \ref{SettingtheStage}). Similarly, the third author conducted the data analysis, and the final results were presented based on the feedback shared by all the authors (see Section \ref{sec:Results}). Finally, all the authors were invited to participate in the brainstorming sessions to discuss the core findings and draw concrete conclusions. 
