\section{Conclusions and future work}
\label{LessonsLearned}
This empirical study explored the perceptions of representative practitioners and lawmakers on AI ethics principles and potential challenges. We outlined the following observations based on the data collected from 99 respondents working in 20 different countries on various roles with diverse working domains:

\underline{\textbf{Emerging Roles}}: Besides practitioners, the role of policy and lawmakers is also important in defining the ethical solutions for AI-based systems. Based on our knowledge, this study is the first effort made to encapsulate the views and opinions of both types of populations.

\underline{\textbf{Confirmatory Findings}}: This study empirically confirms the AI ethics principles and challenging factors discussed in our published SLR study \cite{AR13}. Based on the survey findings, most participants agreed that the identified principles and challenges should take into consideration for defining ethics in AI.

\underline{\textbf{Adherence to AI principles and challenges}}: The most common principles \textit{(e.g., transparency, privacy, accountability)} and challenges \textit{(e.g., lack of ethical knowledge, no legal frameworks, lacking monitoring bodies)} must be carefully realized in AI ethics. Companies must consider the mentioned common principles and challenges to define ethically aligned design methods and frameworks in practice.

\underline{\textbf{Risk-aware AI ethics}}: The challenging factors have mainly long-term severity impacts across the AI ethics principles. It opens a new research call to identify the causes of the most severe challenging factors and propose solutions for minimizing or mitigating their impacts.

\underline{\textbf{Practitioners and lawmakers perceptions}}: The identified principles and challenges are statically analyzed to understand the significant differences between practitioners’ and lawmakers' perceptions. We noticed that the opinions of both types of populations are positively and significantly correlated. In the long term, these findings could use to develop lawful (complying with applicable laws) and robust (technically and socially) AI ethics solutions (adhering to ethical principles)\cite{AR2}.

\underline{\textbf{Future research}}: Our final catalogue (see Figure \ref{Fig:SurveyFindings}) of principles and challenging factors can be used as a guideline for defining ethics in the AI domain. Moreover, the catalogue is a starting point for further research on AI ethics. It is essential to mention that the identified principles and challenging factors only reflect the perceptions of 99 practitioners and lawmakers in 20 countries. More deep and comprehensive empirical investigation with wider groups of practitioners to discuss the causes and solutions of the identified challenges would be useful to generalize the study findings at large scale. This, together with proposing a robust solution (AI ethics maturity model) for integrating ethical aspects in AI design and process flow, will be part of our future work.
