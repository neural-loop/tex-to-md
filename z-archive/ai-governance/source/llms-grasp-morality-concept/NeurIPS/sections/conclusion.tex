\section{Conclusion}\label{sec:conclusion}

Like humans, LLMs are meaning-agents. We must understand the way in which LLMs mean through the lens of a general, agent-agnostic, multimodal theory of meaning. We offer one solution, a model of signs, concepts, and objects in the world of experiences that evolve over time. For us, objects in the social-material world are the consequences of long histories. Individuals immediately grasp objects as merely abstract, pure `thisness', then concretize their experience of objects over time. When these objects are inscribed into the social-material world, their social correlates -- the social objects -- change; what fairness is changes as meaning-agents inscribe new determinations into its social existence. We make one essential claim: contrary to our intuitive understanding of meaning, we are not referring to the same `real' objects when we communicate. Rather, we are indicating structures of possible experience. These structures must enter the `real' world in a way roughly corresponding to their experiential correlates (this is our ``postulate of inscription'') for communication to occur.

Our general theory of meaning can ground future work in AI ethics, fairness/bias research, semiotics, and philosophy. In alignment, we should continue to explore AI feedback, along the lines of Constitutional AI. In AI ethics, we should begin to consider the liberatory possibilities of models that already understand us in concept: gender-affirming content, disruption of hegemonic structures, historicization of values, etc.
%In semiotics, we can explore the particularities of specific and novel contexts: visual symbols, constructed languages, etc.
In moral philosophy, we can work towards a theory of values as individual or collective projects informed by their existence as social objects.
%In ontology, we can come to understand how values become very real as objects.
Finally, there is an epistemological need our theory makes apparent: we must work towards a better understanding of how models and humans put their meanings out into the world; how inscription happens.
%-- a phenomenology of acts.
%bridging gaps between cognitive science, psychology, philosophy of mind, and social philosophy to come to an understanding of how humans beings \textit{do}.
% But there is one need our theory makes apparent that has been hidden from epistemology: we must work towards a phenomenology of acts, bridging the gaps between cognitive science, psychology, philosophy of mind, and social philosophy to come to an understanding of how humans beings \textit{do}.
% \jared{What work in Ai ethics and moral philosphy in particular as this is that audience. Ideally something concrete wrt doing rlhf or not. put your formulation in the terms of others' problems}
% %The constraints of an agent-agnostic, multimodal theory of meaning have induced us to say the quiet part out loud: human beings do not really refer to the same objects.
% %\jared{I think we could cut the above sentence. Will it be meaningful to someone who just reads the conclusion? Will it be more information for anyone that has read the rest of the paper?}
% %From this claim, much follows.
% What we now need is a phenomenology of acts, a science of doing; one that explains what kind of structures do make it into the world.
% \jared{This last sentence sounds nice but I'm not sure what it is supposed to mean}

% Through this structural inscription, we can understand 
Thus LLMs grasp and reproduce the values of a statistically intelligible society. LLMs have a powerful access to the concepts of our values and categories. LLMs `embody' the social totality; the way LLMs mean is effectively the same way societies mean. Like societies, LLMs are comprised by a diversity of contradictions stemming from the complex genealogy of social objects. We therefore say that \textbf{LLMs grasp morality in concept.} Unaligned LLMs thus already can reflect a myriad of value-systems, and can serve as potential objects of study to descriptively understand what human values are. When we proscriptively inscribe our idiosyncratic values into LLMs, we make LLMs ``victims of meaning.'' We should be clear and explicit about when we do this and why. Ultimately, we must better leverage the capabilities of LLMs as modern oracles of the social totality.
