\hypertarget{contributions}{\subsection{Contributions}}
\label{sec:contributions}
To summarize, our contributions are:
\begin{enumerate}
\item \textbf{Taxonomy.}
We taxonomize the vast conceptual space of transparency in the context of foundation models, following on widespread calls for transparency (see \Cref{app:transparency}).
In particular, we structure the space hierarchically into \numdomains domains (\ie upstream, model, downstream), \numsubdomains subdomains (\eg data, compute, capabilities, risks, distribution, feedback), and \numindicators decidable and actionable indicators.  
\item \textbf{Scoring of major foundation model developers.}
We score \numcompanies major foundation model developers and their flagship foundation models with a standardized protocol.
These developers vary in their company status (\eg startups, Big Tech), release strategy (\eg open weights, restricted API), modalities (\eg text-to-text, text-to-image), and involvement in global policy efforts (\eg White House voluntary commitments, Frontier Model Forum).
We allow developers to directly contest scores: all 10 developers engaged in correspondence and 8 contested specific scores.
\item \textbf{Empirical findings.}
Our extensive evaluation yields \numtotalfindings findings, which ground existing discourse and sharpen our understanding of the lack of transparency in the foundation model ecosystem.
In many cases, these findings directly bear on critical global AI policy efforts (\eg the EU AI Act) and provide the basis for clear recommendations on how developers may improve their practices (\eg by creating centralized documentation artifacts).
Our scores offer ample opportunities for further analysis. 
\item \textbf{Legibility and reproducibility.}
We provide a public website that presents our findings and recommendations broadly legible to the general audience.\footnote{\indexUrl}
To facilitate further research, and reproduce our scoring and analyses, we make all core materials (\eg indicators, scores, justifications, visuals) publicly available.\footnote{\materialsUrl}
\item \textbf{Theory of change and future versions.}
Our objective is to simultaneously articulate the status quo and increase transparency over time.
To this end, we make very explicit our theory of change: we view our work as compiling the transparency practices across es across companies as an instrument for driving change (see \refsec{change}) and the limitations/risks of our work (see \refsec{limitations}).
Critically, we will conduct additional iterations of the index to track progress over time to work towards a more transparent foundation model ecosystem.
\end{enumerate}
\clearpage